\documentclass[12pt, %font size
a4paper, %paper type
twoside, % two sided printing
openright, % start new chapter on right side only ( inserts blank pages )
abstract=on, % Use an abstract
DIV=11,      % This parameter organizes the borders (detailed explanation at http://texdoc.net/texmf-dist/doc/latex/koma-script/scrguide.pdf
BCOR=8mm]{scrbook} % BCOR sets the space, used by the type of  book. (e.g. glued, hard cover..)
%scrreprt is used for larger texts with chapters (Master or Bachelor Thesis)
%scrartcl is used when there are no chapters ( for smaller paper)


\usepackage[utf8]{inputenc}
\usepackage[english]{babel} % sets up english hyphenation
\usepackage{csquotes} % for language-dependent quotes in biblatex
\usepackage[unicode=true]{hyperref} % enables use of metadata for pdfs and hyperlinks within a document
\usepackage[natbib,maxnames=2,maxbibnames=100,style=authoryear-comp,uniquename=full,firstinits,doi=false,backend=biber,backref,hyperref]{biblatex} % advanced bibliography support
\usepackage[usenames,dvipsnames,hyperref]{xcolor} % enables more advanced color support for hyperref
\hypersetup{colorlinks=true, %flag for prints
    hidelinks,  % this option would hide links for the print version of your thesis
    linkcolor=red!35!black,    %definition of the link color
    citecolor=green!35!black,  %definition of the cite color
    urlcolor=magenta!35!black, %definition of the url color
    %pdfauthor=, % Optional: Specify the author of the pdf
    %pdftitle=   % Optional: Specify the title within the pdf
}      
\usepackage{subfiles} %This package is used for subfiles
\usepackage{tabu}     % provides advanced tables
\usepackage{array,multirow}
\bibliography{thesis.bib}% Include all the bibliography files
\usepackage{booktabs} % enables reference bookstyle tables
\usepackage[format=plain, labelfont=bf]{caption}
\usepackage[capitalize,noabbrev]{cleveref}
\usepackage{subcaption} % enables use multiple figures in a figure
\captionsetup{compatibility=false}
\usepackage{eurosym} %includes the euro symbol 
\usepackage{enumitem} % allows customization of enumeration and itemize environment
\usepackage{graphicx} % enables loading of graphics
\usepackage{tikz} % drawing vector graphics in latex
%\usepackage{parskip} %alternatively parskip replaces paragraph indentation by increased in-betweeen-paragraph linespacing 
\usepackage{setspace} % helps setup line spacing
%\onehalfspacing % increases linespacing to one and half
\usepackage{placeins} % provides FloatBarrier
%\usepackage[miktex]{gnuplottex} % gnuplot within latex. May be obsolete with pylab.
\usepackage[ruled,vlined]{algorithm2e} %algorithm package
\linespread{1.1} % Definition of the linespread
\usepackage[tbtags]{mathtools}
\DeclareMathOperator*{\somefunc}{somefunc}

%tikz helps to draw nice pictures with a lot of effort for advanced users
\usetikzlibrary{positioning,shapes,shadows,arrows, backgrounds}
\usepackage{verbatim}
\usepackage{tikz-3dplot}


%some definitions for the cref package
\crefname{algocf}{Algorithm}{Algorithms}
\crefname{table}{Table}{Tables}
\crefname{chapter}{Chapter}{Chapters}
\crefname{equation}{Equation}{Equations}
\crefname{section}{Section}{Sections}


\tikzset{
    tri/.style={
        draw,
        shape border rotate=90,
        isosceles triangle,
        isosceles triangle apex angle=60,
        node distance=1cm,
        minimum height=4em
    }
}



\begin{document}
    \frontmatter

    \begin{titlepage}
        \begin{center}

            % Upper part of the page. The '~' is needed because \\
            % only works if a paragraph has started.
            \includegraphics{img/logo2}\\[1cm]

            \textsc{\LARGE Rheinische\\[5mm] Friedrich-Wilhelms-Universität Bonn}\\[1.5cm]

            \textsc{\Large Master thesis}\\[1.5cm]

            % Title
            { \Large \bfseries Basic \LaTeX \, Template }\\[1.4cm]

            % Author and supervisor
            \begin{minipage}[t]{0.4\textwidth}
                \begin{flushleft} \large
                    \emph{Author:}\\
                    Max \textsc{Mustermann}
                \end{flushleft}
            \end{minipage}
            \begin{minipage}[t]{0.5\textwidth}
                \begin{flushright} \large
                    \emph{First Examiner:} \\
                    Prof.~Dr.~John \textsc{Doe} \\[0.5cm]
                    \emph{Second Examiner:} \\
                    Prof.~Dr.~John~\textsc{Doe} \\[0.5cm]

                    \emph{Advisor:} \\
                    John \textsc{Doe} \\[0.5cm]
                        %\emph{Abteilung:} \\
                        %Autonome Intelligente Systeme
                \end{flushright}
            \end{minipage}

            \vfill

            % Bottom of the page
            {\large Submitted:\hspace{1cm} \today}

        \end{center}
    \end{titlepage}

    \pagestyle{headings}  % switches on printing of running heads
    \title{Basic \LaTeX \, Template}
    %\subtitle{Master thesis}
    \author{Tobias Hartmann\\ \begin{minipage}{8cm}\centering \small Friedrich-Wilhelms-Universität Bonn\\ \small (group)\end{minipage}}

    \vspace{4cm}

    \cleardoublepage
    \thispagestyle{empty}
    {\noindent%
        \huge{\textbf{\textsf{Declaration of Authorship}}}
    }
    \vspace{2cm}
    \begin{flushleft}
        \noindent%
        I declare that the work presented here is original and the result of my own investigations.
        Formulations and ideas taken from other sources are cited as such.

        It has not been submitted, either in part or whole, for a degree at this or any other university.
    \end{flushleft}

    \vspace{8cm}
    \noindent%
    \rule[1em]{8em}{0.5pt}  \hfill \rule[1em]{8em}{0.5pt}\\ % This prints a line to write the date
    Location, Date \hfill Signature\\


    \cleardoublepage


    \chapter*{Abstract}
    \thispagestyle{empty}
    Describe your approach and results shortly in the abstract.
    The abstract should really already tell the reader what to expect.
    Do not try to  build suspension, this is not a Holywood  movie, it is a
    (your!) scientific thesis.  The abstract  is usually the last thing you
    write, even if it is the first thing you read here.


    \newpage
    \pagenumbering{roman}
    \tableofcontents
    \listoffigures
    \listofalgorithms

    \newpage
    \mainmatter
    \subfile{content/introduction}

    \chapter{Language}

    \section{I or We?}

    Use the ``we'' form, even when  you're writing alone.  Imagine it to be
    the reader and  you, who are going through the  text together.  It gets
    you closer to your audience.

    \section{Quotes, and Emphasis}
    Note that double  "quotes" are not typeset  correctly.  English quotes 
    should look like ``this'', while  German quotes are typeset like \glqq 
    this\grqq.   Note the  difference.   Some editors  insert the  correct 
    quotes automatically for you.                                          

    Use  emphasis  sparingly.  It  clutters  the  text.  In  general,  use 
    \emph{italics}, which is the one standing out the least.  In captions, 
    it is sometimes useful to use  \textbf{bold} for the signal words left 
    and right, top and bottom: There, you want them to pop out.

    \section{Line Noise and Colloquial Expressions}

    Do  not use  fill words  which do  not carry  meaning.  Try  to be  as 
    specific as  possible.  This does  not mean  as short as  possible, it 
    means that you should have something to say when you write something.
    If you don't know what you want to say, think again, \emph{then} write.

    Don't use colloquial expressions.  In particular, don't use any of the 
    short forms  \emph{don't, aren't,  isn't, you're, we're,  \ldots}.  Do 
    not\footnote{this is much better!} use  ellipsis (\ldots) in the text, 
    as it looks like you're trailing off.                                  

    \chapter{Comments on Writing in LaTeX}

    Lots of people complain about \LaTeX{} standard margins.  Typographers 
    say: You shouldn't have  more than about 70 characters  per line (when 
    using single  spacing).  Otherwise your  eyes have trouble  jumping to 
    the  beginning  of the  next  line.   This  limits your  column  width 
    severely, unless you  use a big font.  A thin  column width results in 
    large margins.  This is (aside  from portability) the reason why books 
    are smaller than A4 paper.

    If you want to toy  with margins nevertheless, consider this: When you 
    open  your  double-sided  printed  thesis, the  white  region  in  the 
    left/right  and  center  region  should have  equal  widths.   Outside 
    margins are larger than the margin inside.  Consequentially, on a left 
    page, left  margins are larger  than right (i.e. center)  margins, and 
    on  a  right page,  right  margins  are  larger  than the  left  (i.e. 
    center) ones. The  margin on the bottom  should be larger than  on the 
    top.  These  rules are  automatically used  by the  KomaScript classes 
    (scrartcl, scrbook, scrrprt).  Do not meddle with them.
    
    To   meddle  with   the   margins,   use  the   DIV   option  of   the 
    \verb+documentclass+  command.    Larger  values  result   in  smaller 
    margins.  If you  want to bind your thesis, make  sure to include some 
    space for it using BCOR.  Ask the  print shop how much to use for your 
    specific binding needs.  Never, ever, use the \verb+geometry+ package. 

    \section{Front, Main and Back Matters}

    Titlepage, lists of figures, table  of contents page etc. are numbered 
    in  roman letters.   The introduction  should start  with page  $1$ in 
    arabic  numerals.   The document  class  \verb+scrbook+  does this  by 
    default,  when \textbackslash  frontmatter, \textbackslash  mainmatter 
    and \textbackslash backmatter are placed properly.                     


    \section{Paragraphs}
    \label{sec:paras}

    Paragraphs are separated  by empty lines in the  \LaTeX{} code.  Never 
    use  the double  backslash for  this purpose.   \LaTeX{} inserts  line 
    breaks  preferably between  paragraphs.  A  double backslash  lets you 
    stay in the  same paragraph, so \LaTeX{} does not  know where it would 
    be good  to insert page  breaks, figures, tables, algorithms,  and the 
    like.                                                                  

    Many people find  the indentation of the first line  in paragraph odd. 
    Again, look into a favorite book  or newspaper of yours and you'll see 
    that this is used everywhere.  Note that it allows you on the top of a 
    new page to  see whether the line  is the first of a  new paragraph or 
    the continuation of an old paragraph. 

    \section{Math}

    Formulas are part of  the text, try to use them  like other objects and
    be sure to include punctuation to keep the text flowing.  For example,
    the pythagorean theorem,
    \begin{align}
        a^2 + b^2 = c^2,
    \end{align}
    is typically taught at school.

    Other random notes:

    \begin{itemize}
        \item
    Note that  as in \cref{sec:paras},  before and after a  formula, empty 
    lines will  add space  and indentation, and  possibly page  breaks, or 
    figures.  So  if a formula  is part of a  paragraph, do not  add empty 
    lines around it in the code.                                           

        \item
    Number all equations.  It makes it  easier to refer to them for others,
    even if you do not refer to them in your own text.

        \item
    While we're at  it, do not use the  \verb+equation+ or \verb+eqnarray+ 
    environments, use \verb+align+ instead.

        \item
    Take care to always put all  math in the math-environment, for example 
    the  variable $x$.   Note  that  it is  typeset  differently from  the 
    letter~x.

        \item
    Use  single-letter  names  for  variables   in  math  (this  is  not  a
    programming language).   Function names  may have  multiple characters,
    such as $\tanh(x)$.  Note that they are not typeset in italics.  
    They  usually have  their own  commands, and  you can  define your  own
    functions, such as  $\somefunc(x)$.  If you don't do  this, the kerning
    will  be broken,  since \LaTeX{}  assumes that  you're multiplying  the
    variables  $s$, $o$,  $m$, $e$,  $f$, $u$,  $n$, and  $c$ and  typesets
    accordingly.

    \end{itemize}

    \section{Referencing}
    Use  the \verb+cleverref+  package to  refer  to other  places in  the 
    document,  e.g.  \cref{chap:Introduction}.    The  package  allows  to 
    specify the reference style globally.  Therefore, it helps to refer in 
    a consistent  way.  Otherwise you  may quickly have different  ways of 
    referencing,   e.g.  \emph{Figure},   \emph{figure},  \emph{Fig.}   and
    \emph{fig.}.   The package  also ensures  that there  is no  linebreak 
    before the number of the figure/chapter etc.                           

    \section{Citations}
    There are two ways to cite, which depends on whether the authors are an
    active part of your sentence:
    \begin{itemize}
        \item \citet{muster} showed that ...
        \item This is currently a hot research topic \citep{muster}
    \end{itemize}
    This distinction  is even more important  when you use a  numeric style
    for referencing.

    In  your bibliography,  you should  define common  strings, using  the 
    \verb|@string| syntax.  Otherwise you might  quickly end up with a lot 
    of different  names for  the same conference,  which is  confusing and 
    inconsistent.  (Please have a look at the bibliography file.)          

    \section{Enumerations}

    Short enumerations look ugly in  standard \LaTeX{}, since they include 
    too much  space between the  items.  A ``normal'' enumerate  list with 
    very short paragraphs:                                                 
    \begin{enumerate}
        \item foo bar baz
        \item foo bar baz
        \item foo bar baz
    \end{enumerate}

    A list using enumitem to compress the inter item spaces, looks much better for short items:
    \begin{enumerate}[noitemsep]
        \item foo bar baz
        \item foo bar baz
        \item foo bar baz
    \end{enumerate}

    \begin{figure}
        \centering
        \begin{subfigure}[b]{0.3\textwidth}
            \includegraphics[width=\textwidth]{img/cat.jpeg}
            \caption{Nyan cat}
            \label{fig:cat1}
        \end{subfigure}%
        ~ %add desired spacing between images, e. g. ~, \quad, \qquad, \hfill etc.
        %(or a blank line to force the subfigure onto a new line)
        \begin{subfigure}[b]{0.3\textwidth}
            \includegraphics[width=\textwidth]{img/cat.jpeg}
            \caption{Nyan cat}
            \label{fig:cat2} 
        \end{subfigure}
        ~ %add desired spacing between images, e. g. ~, \quad, \qquad, \hfill etc.
        %(or a blank line to force the subfigure onto a new line)
        \begin{subfigure}[b]{0.3\textwidth}
            \includegraphics[width=\textwidth]{img/cat.jpeg}
            \caption{Nyan cat}
            \label{fig:cat3}  
        \end{subfigure}
        \caption[Nyan cats]{Pictures of Nyan cats}\label{fig:cats}
    \end{figure}


    \section{Figures}
    You should  allow \LaTeX{} to place  the figures where it  wants.  The 
    same  goes  for  tables.   This  is  called  a  \verb+float+,  as  the 
    figure/table floats around.   Look in a well-typeset  book, and you'll 
    notice that figures aren't placed randomly in the text, they're rather 
    always  on the  top  of  the page  if  at  all possible. \LaTeX{}  has 
    options to  let figures float to  \emph{top} (\verb+t+), \emph{bottom} 
    (\verb+b+),   \emph{page  of   floats}   (\verb+p+)  and   \emph{here} 
    (\verb+h+).  The  default is \verb+tbp+ is  ok for most cases,  in any 
    case, never use \verb+h+.  You can use the FloatBarrier command if you 
    want to make sure that a figure is within a specific section.          

    All figures have  a caption from which you can  understand most of the 
    figure content  and its significance.   They are also  referenced from 
    the main text (e.g., see \cref{fig:cats}).  Do not force linebreaks in 
    captions.  Do not deviate from these rules, they are strict.           

    Figures  should  provide a  short  name  for  every figure  in  square 
    brackets: $[$~$]$. If  you do not define a short  name, the whole text 
    of the caption will be displayed, which is usually too long.
    
    Note that \cref{fig:cats} also shows how to use subfigures.



    \section{Tables}
    Never  use vertical  lines in  tables.  See  the documentation  of the 
    booktabs package for an explanation,  or look in a favorite scientific 
    book/magazine to understand that this is a commonplace rule.

    A basic  example, taken  from the booktabs  documentation, is  given in
    \cref{tab::ex}.

    \begin{table}
        \centering
        \begin{tabular}{@{}lrr@{}} 
            \toprule
            \multicolumn{2}{c}{Education}\\ \cmidrule{1-2}
            Major & Duration & Income (\euro)\\ 
            \midrule 
            CompSci & 2 & 12,75 \\ \addlinespace
            MST & 6 & 8,20 \\ \addlinespace
            VWL & 14 & 10,00\\ 
            \bottomrule
        \end{tabular}
        \caption[Table Example]{A basic example from the booktabs package.}
        \label{tab::ex}
    \end{table}

    \subsection{The tabu package}
    The tabu package provides some useful tools for advanced tables.


    \section{Algorithms}
    There are several packages to typeset algorithms.
    We recommend the \verb+algorithm2e+ package.
    In \cref{alg:exp} a simple example from the algorithm2e documentation is given.

    \begin{algorithm}[t]
        \SetAlgoLined
        \KwData{this text}
        \KwResult{how to write algorithm with \LaTeX2e }
        initialization\;
        \While{not at end of this document}{
            read current\;
            \eIf{understand}{
                go to next section\;
                current section becomes this one\;
            }{
                go back to the beginning of current section\;
            }
        }
        \caption{How to write algorithms (Small example from the algorithm2e documentation)}
        \label{alg:exp}
    \end{algorithm}

    \section{Compiling The Document}
    \LaTeX{}  is  a bit  archaic  to  work  with,  since with  every  pass 
    through  your document,  some intermediate  information is  written to 
    files,  which are  then processed  by  other programs.   Keep in  mind 
    that  everytime a  new  pass through  your  document incorporates  new 
    information (e.g. from BibTeX), page breaks may change, which requires
    an  additional pass  through the  document. Thus, you  usually have  to
    compile the document in three steps:
    \begin{enumerate}
        \item Compile the document using xelatex: xelatex thesis
        \item Run biber  (or if  you do  not have  it, bibtex)  to set  up
              bibliography: biber thesis
        \item Run xelatex  twice more to incorporate  bibtex references and
              to get the typesetting right.
    \end{enumerate}

    If  you know  what  you're doing  you  can get  away  with less.   Some
    programs parse the  output of \LaTeX{} to  determine whether additional
    passes are necessary.  Most GUI programs fall in this category, as do
    \verb+rubber+ and \verb+latexmk+.

    \backmatter

    \chapter{Appendix}
    Note that it the missing chapter number,  since it is behind
    the backmatter command.

    \FloatBarrier

    \begin{singlespacing}
        \printbibliography
    \end{singlespacing}

\end{document}

